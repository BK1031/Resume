%%%%%%%%%%%%%%%%%%%%%%%%%%%%%%%%%%%%%%%%%
% Developer CV
%----------------------------------------------------------------------------------------
%	PACKAGES AND OTHER DOCUMENT CONFIGURATIONS
%----------------------------------------------------------------------------------------

\documentclass[9pt]{developercv} % Default font size, values from 8-12pt are recommended
\usepackage{multicol}
\setlength{\columnsep}{0mm}
%----------------------------------------------------------------------------------------
\usepackage{lipsum}  


\begin{document}

%----------------------------------------------------------------------------------------
%	TITLE AND CONTACT INFORMATION
%----------------------------------------------------------------------------------------

\begin{minipage}[t]{0.5\textwidth} 
	\vspace{-\baselineskip} % Required for vertically aligning minipages
	
	{ \fontsize{16}{20} \textcolor{black}{\textbf{\MakeUppercase{Bharat Kathi}}}} % First name
	
	\vspace{6pt}
    \icon{Envelope}{11}{\href{mailto:bkathi@ucsb.edu}{bkathi@ucsb.edu}}
    \hspace{22pt}
    \icon{Phone}{11}{510 945 9684}

    \icon{Github}{11}{\href{https://github.com/bk1031}{github.com/bk1031}}
    \hspace{12pt}
    \icon{LinkedinSquare}{11}{\href{https://www.linkedin.com/in/bk1031/}{/in/bk1031}}
    
\end{minipage}
\hfill
\begin{minipage}[t]{0.5\textwidth}
    \vspace{-18pt}
    \cvsect{Skills}
    \vspace{-6pt}
    
    \begin{minipage}[t]{0.2\textwidth}
        \textbf{Languages:}
    \end{minipage}
    \hfill
    \begin{minipage}[t]{0.73\textwidth}
        C++, Dart, Golang, Java, NodeJS, Python, Swift
    \end{minipage}
    \vspace{5 pt}
    
    \begin{minipage}[t]{0.2\textwidth}
        \textbf{Technologies:}
    \end{minipage}
    \hfill
    \begin{minipage}[t]{0.73\textwidth}
        AWS, Azure, Docker, Firebase, GCP, Git, Kafka, Kubernetes, Pandas, PostgreSQL, PyTorch, RabbitMQ, SingleStore, Snowflake
    \end{minipage}
    
\end{minipage}

%----------------------------------------------------------------------------------------
%	EDUCATION
%----------------------------------------------------------------------------------------
\cvsect{Education}
\begin{entrylist}
    \vspace{-4pt}
    \entry
		{}
		{University of California, Santa Barbara}
		{September 2021 – June 2025}
		{B.S. Computer Engineering
        \begin{itemize}[noitemsep,topsep=0pt,parsep=0pt,partopsep=0pt, leftmargin=10pt]
            \item Relevant Coursework: Circuits and Systems, Computer Architecture, Computer Vision, Data Structures \& Algorithms, Deep Learning, Distributed Systems, Embedded Systems, Machine Learning, Operating Systems
        \end{itemize}}
\end{entrylist}
\vspace{-10pt}

%----------------------------------------------------------------------------------------
%	EXPERIENCE
%----------------------------------------------------------------------------------------
\cvsect{Experience}
\begin{entrylist}
    \entry
        {}
        {Gaucho Racing (UCSB Formula SAE) • Data Lead}
        {September 2021 – Present}
        {\vspace{-8pt}
        \begin{itemize}[noitemsep,topsep=0pt,parsep=0pt,partopsep=0pt, leftmargin=10pt]
            \item Founded and led the Data subteam, building Mapache: an intelligent, real-time telemetry and analytics system for our team's Formula-style electric racecar
            \item Designed an embedded edge stack with Jetson Orin Nano to decode 200+ sensor signals over CAN and execute real-time ML models for thermal prediction and anomaly detection
            \item Created a robust telemetry pipeline that caches data onboard in the event of connectivity loss, and uploads it post-run using GPU-accelerated Parquet conversion and cuDF, ensuring zero data loss
            \item Developed cloud infrastructure using MQTT, AWS, and SingleStore to ingest, store, and analyze high-frequency telemetry for performance diagnostics and tuning
            \item Built a React + TypeScript dashboard with real-time signal visualizations, trip/lap segmentation, and historical comparisons across drivers and configurations
            \item Integrated a natural language interface powered by LLMs, enabling engineers to query telemetry insights conversationally (e.g., “Compare battery temperature rise across laps”)
            \item Collaborated closely with vehicle dynamics and powertrain teams to validate sensor data and translate telemetry insights into mechanical design improvements
            \item Implemented continuous monitoring of system health (Jetson resource usage, connectivity, thermal load), with alerts routed back to the driver via CAN during operation
            \item This system was the first of its kind in the history of Formula SAE, and has collected over 20 GB (over 1B datapoints) of vehicle data
        \end{itemize}}
	\entry
        {}
		{SingleStore • Software Engineering Intern}
		{June 2024 – September 2024}
		{\vspace{-8pt}
        \begin{itemize}[noitemsep,topsep=0pt,parsep=0pt,partopsep=0pt, leftmargin=10pt]
            \item SingleStore is a distributed, high-performance SQL database that unifies transactional and analytical workloads (HTAP) in a single engine
            \item Contributed to SingleStore's integration into Snowflake via Snowpark Container Services, enabling real-time analytics inside Snowflake's data warehouse
            \item Improved onboarding by adding preset cluster configuration options in the Python based UI (Streamlit), streamlining setup for new users
            \item Built and presented a demo simulating a real-time ridesharing app, streaming live driver/rider data and visualizing insights in a React dashboard
            \item Demonstrated up to 50x lower latency and faster queries compared to native Snowflake workloads, highlighting the performance benefits of SingleStore
            \item Presented the demo to enterprise customers and at SingleStore NOW, supporting product launch and early adoption
        \end{itemize}}
    \entry
        {}
		{Axiamatic (Greylock-backed startup) • Software Engineering Intern}
		{June 2023 – September 2023}
		{\vspace{-8pt}
        \begin{itemize}[noitemsep,topsep=0pt,parsep=0pt,partopsep=0pt, leftmargin=10pt]
            \item Built internal tools with React and TypeScript for Admin UI, an internal tool for engineers to manage and monitor the platform
            \item Developed new backend services in Python (FastAPI) for the Admin UI, and deployed them on AWS Fargate
            \item Created a Service Registry Browser to visualize 100+ AWS-deployed microservices and their dependencies, significantly improving visibility across teams and accelerating onboarding for new engineers
            \item Built an On-call Log system to track incidents, document responses, and notify responsible teams via Slack, reducing on-call resolution time and improving cross-team coordination during outages
        \end{itemize}}
	\entry
		{}
		{Pacific Esports League • Software Engineering Intern}
		{June 2022 – September 2022}
		{\vspace{-8pt}
        \begin{itemize}[noitemsep,topsep=0pt,parsep=0pt,partopsep=0pt, leftmargin=10pt]
            \item Built core features for the PEL Portal, a Flutter-based web app used by 1,000+ players to form teams, register for tournaments, and view real-time stats
            \item Designed a custom CMS for league admins to manage tournaments, streamlining team and player management
            \item Implemented a matchmaking system to auto-seed brackets, notify players, and collect results, reducing manual labor by more than 20 hours/week
            \item Set up CI/CD pipelines with GitHub Actions to auto-deploy Go-based microservices to Azure Container Apps
        \end{itemize}}
\end{entrylist}
\vspace{-10pt}

%----------------------------------------------------------------------------------------
%	Projects
%----------------------------------------------------------------------------------------
\cvsect{Projects}
\begin{entrylist}
    \entry
		{}
		{StorkeCentral • 1,500+ Users}
		{storkecentral.app}
        {\vspace{-8pt}
        \begin{itemize}[noitemsep,topsep=0pt,parsep=0pt,partopsep=0pt, leftmargin=10pt]
            \item Created a platform for UCSB students to access important campus information, and scaled to over 1,000 monthly active users
            \item StorkeCentral provide easy access to dining hall menus, campus maps, bus schedules, and more
            \item Users can add their friends, share class schedule information, and receive notifications when friends are in class
            \item Flutter frontend, Go + PostgreSQL + Kubernetes backend
        \end{itemize}}
    \entry
		{}
		{SproutSense}
		{github.com/BK1031/SproutSense}
		{\vspace{-8pt}
        \begin{itemize}[noitemsep,topsep=0pt,parsep=0pt,partopsep=0pt, leftmargin=10pt]
            \item Designed and built SproutSense as part of my senior capstone project: a distributed LoRa-based telemetry system for real-time environmental monitoring across large agricultural fields
            \item Sensor Modules trasmit readings to nearby Base Stations, forming a star topology minimizing node complexity/power and simplifying large-field scaling
            \item Developed custom PCBs for both Sensor Modules and Base Stations, supporting over 100 modules per base station, communicating over 915 MHz LoRa using Listen-Before-Talk (LBT) to reduce collisions
            \item STM32-based Sensor Modules measure temperature, humidity, soil moisture, soil nutrition, and light levels at configurable intervals (30 min to 24 hrs)
            \item ESP32-based Base Stations relay LoRa messages to an MQTT broker, which are then ignested by a Python Flask application where messages are parsed, scaled, and stored in a PostgreSQL database
            \item An exponential weighted moving average (EWMA) is used at each node to estimate channel activity and probabilistically delay transmissions to avoid collisions
            \item React + Typescript frontend enables farmers to visualize historical trends and detect anomalies across the field
            \item Designed for ultra-low power operation and long-term deployment in remote, offline environments, powered by solar with a battery backup lasting over a week
        \end{itemize}}
    \entry
		{}
		{Epic Shelter}
		{github.com/gaucho-racing/EpicShelter}
		{\vspace{-8pt}
        \begin{itemize}[noitemsep,topsep=0pt,parsep=0pt,partopsep=0pt, leftmargin=10pt]
            \item Created Epic Shelter, a high-performance, distributed data backup/migration system
            \item Supports PostgreSQL and MySQL wire compliant databases, as well as SingleStore and Snowflake
            \item Data is backed up from a source database table as a batch of parquet files, backed up to S3, then ingested into a target database table
            \item Orchestrator service allows distributing data movement across multiple worker nodes, and monitoring the status of the backup/migration
            \item cuDF is used to acclerate reading/writing parquet files on GPUs
            \item Parquet files from S3 are directly ingested into the target database when supported (SingleStore, Snowflake)
            \item React + Typescript frontend, Go orchestrator, Python backend (cuDF + Pandas)
        \end{itemize}}
	\entry
		{}
		{Sentinel}
		{github.com/gaucho-racing/Sentinel}
		{\vspace{-8pt}
        \begin{itemize}[noitemsep,topsep=0pt,parsep=0pt,partopsep=0pt, leftmargin=10pt]
            \item Created Sentinel, a central authentication service for Gaucho Racing (UCSB's Formula SAE team)
            \item Keeps track of an active member directory, interfacing with our Discord server to sync subteams and roles
            \item Implemented OAuth 2.0 and OpenID Connect protocols to allow internal tools to authenticate members, as well as serve as an identity provider for external tools
            \item Automated onboarding/offboarding processes for the team, including managing permissions for Google Drive, GitHub, and even SOLIDWORKS CAD Licenses.
            \item React + Typescript frontend, Go + SingleStore backend
        \end{itemize}}
\end{entrylist}
\vspace{-10pt}

%----------------------------------------------------------------------------------------
%	Publications
%----------------------------------------------------------------------------------------
\cvsect{Publications}
\begin{entrylist}
    \entry
		{}
		{Mapache - An Intelligent, Real-Time Telemetry Platform for Formula SAE}
		{}
		{\vspace{-8pt}
        \begin{itemize}[noitemsep,topsep=0pt,parsep=0pt,partopsep=0pt, leftmargin=10pt]
            \item Bharat Kathi, Andrey Otvagin, Jacob Jurek, Austin Chan, Colin Bickel
            \item Accepted to the 2025 International Telemetering Conference (ITC), to be published in their Vol. 60 Proceedings
            \item https://github.com/Gaucho-Racing/Mapache/blob/main/itc25.pdf
        \end{itemize}}
    \entry
		{}
		{Real-Time Environment Monitoring for Sustainable Agricultural Practices}
		{}
		{\vspace{-8pt}
        \begin{itemize}[noitemsep,topsep=0pt,parsep=0pt,partopsep=0pt, leftmargin=10pt]
            \item Julia Chan, Max Cohn, Bharat Kathi, Pablo Sandoval Rivas, Kriteen Shrestha, Yogananda Isukapalli
            \item Accepted to the 2025 International Telemetering Conference (ITC), to be published in their Vol. 60 Proceedings
            \item https://github.com/BK1031/SproutSense/blob/main/itc25.pdf
        \end{itemize}}
\end{entrylist}
\vspace{-10pt}

\end{document}
